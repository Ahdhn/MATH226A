\section*{Problem No.1} \label{sec:prob1}

An iteration of Newton's method is 
\[
x_{k+1} = x_{k} - \frac{f(x_{k})}{f^{\prime}(x_{k})} 
\]

We can subtract $x_{*}$ from both sides and rearrange. We then get
\begin{equation}
\frac{x_{k+1} -x_{*}}{x_{k} -x_{*}} = 1 - \frac{f(x_{k})}{f^{\prime}(x_{k}) (x_{k}-x_{*})} 
\end{equation}


We can use Taylor expansion to express $f(x_{k})$ and $f^{\prime}(x_{k})$ in terms of $f(x_{*})$ and higher order terms as follows
\[
f(x) = f(x_{*}) + f^{\prime}(x_{*})(x-x_{*}) + \frac{f^{\prime \prime}(x_{*})}{2}(x-x_{*})^{2}+HOT 
\] 
where $HOT$ stands for \emph{higher order terms}. Since $f(x)=f^{\prime}(x)=0$, we get
\begin{equation}
f(x) = \frac{f^{\prime \prime}(x_{*})}{2}(x-x_{*})^{2}+HOT
\end{equation}
We can differentiate (2) once about $x$. We get
\begin{equation}
f^{\prime}(x) = f^{\prime \prime}(x_{*})(x-x_{*})+HOT
\end{equation}

We can then evaluate (2) and (3) at $x_{k}$ and substitute in (1) to get
\[
\frac{x_{k+1} -x_{*}}{x_{k} -x_{*}} = 1 - \frac{  \frac{f^{\prime \prime}(x_{*})}{2}(x_{k}-x_{*})^{2}+HOT }{ (f^{\prime \prime}(x_{*})(x_{k}-x_{*})+HOT) (x_{k}-x_{*})}
\]
\[
\frac{x_{k+1} -x_{*}}{x_{k} -x_{*}} = 1 - \frac{1}{2} \left( \frac{1}{1 + HOT*(f^{\prime \prime}(x_{*})(x_{k}-x_{*}))^{-1}}\right)
\]
One can argue that the term $HOT*(f^{\prime \prime}(x_{*})(x_{k}-x_{*}))^{-1}$ will eventually go to zero with $k \to \infty$ since the solution gets closer and closer to $x_{*}$. Thus, by taking the limits as $k\rightarrow \infty$ in the last expression, the higher order terms will go to zero and we get 
$$
\lim_{k \to \infty} \frac{x_{k+1} -x_{*}}{x_{k} -x_{*}} = \lim_{k \to \infty} \frac{e_{k+1}}{e_{k}} = \frac{1}{2} \qquad \qquad \qquad \blacksquare
$$