\section*{Problem No.1} \label{sec:prob1}

\paragraph{Part (a):} 
The following MATLAB code computes the exponential function $f(x)=e^x$ using the truncated series up to $n$ terms. 
\begin{lstlisting}
	function approx_exp = ApproxExpFunc(x, n)
    %Approximate the exponential using truncated series    
    approx_exp = 1;  %first term is 1 (x^(0)/0!)
    j_factorial = 1; %start with factorial of 0
    for j=1:n
        j_factorial = j_factorial*(j);
        approx_exp = approx_exp + (x^(j))/j_factorial;    
    	end
	end
\end{lstlisting}

With $n=10$ and $x=\pm 5$, the relative error is of order $10^{-11}$. However for $x= 30\pi$, it is $\approx 1$ and for $x=-30\pi$, it is $10^{54}$. Using higher value for $n$ (i.e., $n=40$), it helps decreasing the relative error for $x=\pm 5$ to be $ <10^{-16}$, it does not affect relative error of  $x= 30\pi$ but it increases the relative error for  $x=-30\pi$ to be $10^{71}$. 
 
The reason behind this is that the problem is ill-conditioned. The condition number can be calculated as

$$
\kappa(x) = \frac{||J||}{\frac{||f(x)||}{||x||}}
$$
$$
\kappa(x) =\frac{||\sum_{j=0}^{n} \frac{jx^{j-1}}{j!} ||}{\frac{||\sum_{j=0}^{n} \frac{x^{j}}{j!}|| }{||x||} }
$$
If we choose to take $|| \cdot||_{\infty}$, then we would end up with $\kappa(x)$ as a function of $x$ which explains why the error gets larger with large values of $x$ for the same $n$ value;

\paragraph{Part (b):} Using remainder theorem for Taylor series, we can expect the error to be $<10^{-13}$ only when $abs(x) < 0.33$. This is derived by solving the following:

$$
E(x) = \frac{ abs(e^x - R_{n,x})}{e^x} = \frac{e^x - \sum_{j=0}^{n}}{e^x} <10^{13}
$$
where $n=10$. Using the fact that $e^{x} = e^{x/2 + x/2}=e^{x/2}  e^{x/2}$, we can designed an algorithm that runs recursively by splitting $x$ into half and only runs the truncated series when $abs(x) < 0.33$. The following MATLAB code implements such an algorithm. As a safe guard, we used $abs(x)<0.2$. This gives a relative error for inputs $-100 \leq x  \leq 100$ to be always $<10^{-13}$.

\begin{lstlisting}
function accurate_exp = AccurateExpFunc(x,n)
    %More accurate approximation for the exp function
    %it exploites the fact that exp(x) = exp(x/2+x/2) = exp(x/2)*exp(x/2)
    %with in mind that ApproxExpFunc() give good approximation 
    %(relative err < 10^-13) for x<0.2
    
    if abs(x) < 0.2
       accurate_exp = ApproxExpFunc(x, n);
    else 
        accurate_exp = AccurateExpFunc(x/2, n);
        accurate_exp = accurate_exp*accurate_exp;
    end
end
\end{lstlisting}

